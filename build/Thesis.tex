
%% ----------------------------------------------------------------
%% Thesis.tex -- MAIN FILE (the one that you compile with LaTeX)
%% ---------------------------------------------------------------- 

% Set up the document
\documentclass[a4paper, 11pt, oneside]{Thesis}  % Use the "Thesis" style, based on the ECS Thesis style by Steve Gunn
\graphicspath{{Figures/}}  % Location of the graphics files (set up for graphics to be in PDF format)

% Include any extra LaTeX packages required
\usepackage[square, numbers, comma, sort&compress]{natbib}  % Use the "Natbib" style for the references in the Bibliography
\usepackage{verbatim}  % Needed for the "comment" environment to make LaTeX comments
\usepackage{vector}  % Allows "\bvec{}" and "\buvec{}" for "blackboard" style bold vectors in maths
\usepackage{tabulary}
\usepackage{tikz}
\usepackage{gantt}
\usepackage{rotating}
\usepackage{pdflscape}
\usepackage{Sweave}
\usepackage{graphics}
\usepackage{float}
\usepackage{hyperref}
\usepackage{placeins}


\newcommand{\includecode}[2][C]{\lstinputlisting[caption=$2, escapechar=, style=custom$1]{$2}}

\lstdefinestyle{customc}{
  belowcaptionskip=1\baselineskip,
  breaklines=true,
  frame=L,
  xleftmargin=\parindent,
  language=C,
  showstringspaces=false,
  basicstyle=\footnotesize\ttfamily,
  keywordstyle=\bfseries\color{green!40!black},
  commentstyle=\itshape\color{purple!40!black},
  identifierstyle=\color{blue},
  stringstyle=\color{orange},
}

\lstdefinestyle{customasm}{
  belowcaptionskip=1\baselineskip,
  frame=L,
  xleftmargin=\parindent,
  language=[x86masm]Assembler,
  basicstyle=\footnotesize\ttfamily,
  commentstyle=\itshape\color{purple!40!black},
}

\lstset{escapechar=@,style=customc}



\hypersetup{urlcolor=blue, colorlinks=true}  % Colours hyperlinks in blue, but this can be distracting if there are many links.

%% ----------------------------------------------------------------
\begin{document}
\frontmatter      % Begin Roman style (i, ii, iii, iv...) page numbering

EPIDEV

 \textbf{TipIt Development Proposal}

 Mobile Application to facilitate on the spot tipping

October 2014

By Keith Brown




\pagestyle{fancy}  %The page style headers have been "empty" all this time, now use the "fancy" headers as defined before to bring them back


%% ----------------------------------------------------------------
\mainmatter
\pagestyle{fancy}

\chapter{Functionality}
\label{Functionality}
\lhead{ \emph{Functionality}}

TipIt functionality can be broken down into five separate groups.

\begin{itemize}
\item  Registration of accounts and bank / credit card details
\item  Sending Payments
\item  Receiving Payments
\item  Transaction Reporting
\item  QR code tip flow
\item  Administration Utilities
\end{itemize}

\section{Registration}

Typically a web application requires a user to register before they may use the service. TipIt is no different and will require the standard sign up process. This can either be done directly in the application or on the website.

However, credit card details must be taken before a tip can be made. It would be convenient if we could record their credit card details on registration so that a unnecessary step is not introduced. \textbf{Note}: Do we want to allow for multiple credit cards? For example someone might have a business and a personal credit card.

Bank account details are required if payments are to be received. It is unlikely that all users will require need to receive payments so registration of a bank account can be optional. A prompt can be shown on the dashboard requesting to add a bank account to receive funds.

It is possible for the user to receive payments and to keep it in their virtual `TipIt` Bank account. Once they add their bank account the funds can be transferred.


So the fields that must be entered are:

\begin{itemize}
\item  Unique user name
\item  First and Last Name
\item  Company Name (optional) - will be displayed instead of name if entered
\item  Email Address
\item  Password
\item  Credit card details
\end{itemize}


\section{Sending Payments}

Users of the application may issue tips to other users. This can be done by searching for a unique username - or perhaps email? They also may scan a QR code to bring up that users details. Any amount over a certain value can be entered. A confirmation box must be shown before committing to the tip.


Sending tips are facilitated by using Pin Payments /charges API request. The user must have a valid credit card previously registered with the Pin Payment /customers API. Payments take 7 days to be cleared, and at that point it can be transferred to the recipient`s bank account.

However, the documentation does not state how we can tell when a payment has successfully transferred. I am assuming that we need to poll the /charges/token and check the `Status\_message` but it is not stated.
\section{Receiving Payments}


Pin Payments API provides a /transfers request to move money to a bank account. A bank account must be previously registered with the /recipients API.



 \textbf{Note}: We must register for the private beta to get access to this feature.

 \textbf{Note}: It is unclear if this will incur the 2.6\% transaction fee on top of the previous 2.6\% fee from the credit card charge.

 \textbf{Note}: It is unclear how long it takes to transfer the funds from Pin Payments to the bank account.

\section{Transaction Reporting}

Transaction reporting is actually handled completely by us.

There are two actions that trigger an update to the transaction list:

\begin{enumerate}
\item  When someone tips with their credit card, a pending payment is issued
\item  When the pending payment is cleared by Pin Payment, the payment is marked as complete
\end{enumerate}

It would be nice to offer a graph to people that are receiving tips on their transaction list.
\section{QR code tip flow}
\section{Administration Utilities}
\chapter{Technologies}
\label{Technologies}
\lhead{ \emph{Technologies}}

Although TipIt is a mobile application, it falls within a web technology / software as a service model.

\begin{itemize}
\item  Cordova - Cross platform mobile application wrapper
\item  Pin Payments - Secure Payment gateway
\item  AWS - Amazon Web services
\item  EPIDEV Web Application Toolchain
\end{itemize}


\section{Cordova}
\section{Pin Payments}
\section{Amazon Web Services}
\section{EPIDEV Web Application Toolchain}
\chapter{Limitations}
\label{Limitations}
\lhead{ \emph{Limitations}}

Delay of payments
Cordova limitations
Scalability

\chapter{Schedule}
\label{Schedule}
\lhead{ \emph{Schedule}}
\chapter{Costs and Expenses }
\label{Costs and Expenses }
\lhead{ \emph{Costs and Expenses }}





\end{document}  % The End
%% ----------------------------------------------------------------
